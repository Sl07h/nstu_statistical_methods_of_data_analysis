\documentclass[12pt, a4paper]{article}
\usepackage[russian]{babel}
\usepackage{fontspec}
\setsansfont{Calibri}
\setmonofont{Consolas}
\setmainfont[
    Ligatures=TeX,
    Extension=.otf,
    BoldFont=cmunbx,
    ItalicFont=cmunti,
    BoldItalicFont=cmunbi,
]{cmunrm}
\usepackage{polyglossia}
\setdefaultlanguage{russian}
\setotherlanguage{english}


\usepackage{geometry}
\usepackage{pgfplotstable}
\usepackage[utf8]{inputenc}
\usepackage[english]{babel}
 
\setlength{\parindent}{0.7em}
\setlength{\parskip}{0.7em}
\geometry{
margin=2cm
}

% Создаем команду, чтобы переносить текст на новую строку внутри таблицы
\newcommand{\tcell}[2][l]{\begin{tabular}[#1]{@{}c@{}}#2\end{tabular}}

\usepackage{indentfirst}

\usepackage{arydshln}
\usepackage[fleqn]{amsmath}
\usepackage{xfrac}
\usepackage{esint}
\usepackage{amssymb}
\usepackage{mathbbol}
\usepackage[T1]{fontenc}
\usepackage{mathtools}
\usepackage{color}
\usepackage{ulem}
\usepackage{tabu}
\usepackage{multirow}
\usepackage{rotating}
\usepackage{enumitem}
%\usepackage{booktabs}
\usepackage[outline]{contour}
\contourlength{1.2pt}

\usepackage{tikz}
\usepackage{graphics}
\usepackage{xcolor}

\usepackage{pgfplots}
\usepackage{pgfplotstable}

\usepackage[at]{easylist}

\DeclareMathOperator{\sign}{sign}



%-------------------------------------------------------------------------------
%-------------------------------------------------------------------------------
%-------------------------------------------------------------------------------
\newcommand{\insertTitle}[9]{
\begin{titlepage}
	\begin{center}
    	\large
		Министерство науки и высшего образования Российской Федерации
		
		Новосибирский государственный технический университет
		%\vspace{0.25cm}
		\vfill
		{\textbf #1}
		
		Лабораторная работа №#2
		\vfill
	\end{center}
	
	\begin{tabular}{ m{7em}  m{7em} }
	Факультет: & ФПМИ \\ 
	Группа: & #3 \\  
	Студенты: & #4 \\
			   & #5 \\
			   & #6 \\
			   & #7 \\	
	Вариант: & #8
	\end{tabular}
	\vfill

\begin{center}
Новосибирск

#9
\end{center}
\end{titlepage}
}




\newcommand{\inputTable}[1]{
\noindent{\pgfplotstabletypeset[
 	fixed,
    fixed zerofill,
    precision=3,
	columns/theta_l/.style	={column type={|c|}, column name={$\hat{\theta}_j - t_{\alha / 2, f_R} \sigma(\hat{\sigma}_j)$}},
	columns/theta/.style		={dec sep align={c|}, column type/.add={}{|}, column name={$\hat{\theta}_j$} },
	columns/theta_r/.style	={dec sep align={c|}, column type/.add={}{|}, column name={$\hat{\theta}_j + t_{\alha / 2, f_R} \sigma(\hat{\sigma}_j)$} },
	columns/F/.style			={dec sep align={c|}, column type/.add={}{|}, column name={$F$} },
	columns/F_t/.style		={dec sep align={c|}, column type/.add={}{|}, column name={$$F_{\alpha,q,n-m}$$} },
	columns/significance/.style	={string type, column type/.add={}{|}, column name={$\text{значимость параметра}$} },
	every head row/.style={before row=\hline,after row=\hline},
	every last row/.style={after row=\hline},
]{#1}}
}




\usepackage[utf8]{inputenc}
\usepackage{listings}
\usepackage{color}
 
\definecolor{codegreen}{rgb}{0,0.6,0}
\definecolor{codegray}{rgb}{0.5,0.5,0.5}
\definecolor{codepurple}{rgb}{0.58,0,0.82}
\definecolor{backcolour}{rgb}{0.85,0.85,0.85}
 
\lstdefinestyle{mystyle}{
	backgroundcolor=\color{backcolour},   
	commentstyle=\color{codegreen},
	keywordstyle=\color{blue},
	basicstyle=\fontsize{10}{12}\selectfont\ttfamily,
   	numberstyle=\tiny\color{codegray},
	stringstyle=\color{codepurple},
    	breakatwhitespace=false,         
	breaklines=true,                 
	captionpos=b,                    
	keepspaces=true,                 
	numbers=left,                    
	numbersep=5pt,                  
	showspaces=false,                
	showstringspaces=false,
	showtabs=false,                  
	tabsize=4
}
\lstset{ style=mystyle}


\newcommand{\inputTwoImages}[2]{
\begin{figure}[htbp!]
    \noindent
        \includegraphics[width=0.48\textwidth]{#1}
        \includegraphics[width=0.48\textwidth]{#2}
\end{figure}
}


\newcommand{\inputThreeImages}[3]{
\begin{figure}[htbp!]
    \noindent
        \includegraphics[width=0.32\textwidth]{#1}
        \includegraphics[width=0.32\textwidth]{#2}
        \includegraphics[width=0.32\textwidth]{#3}
\end{figure}
}



\newcommand{\myCodeInput}[3]{
{\bf Файл #2}
\lstinputlisting[language=#1]{#3}
}



%-------------------------------------------------------------------------------
%-------------------------------------------------------------------------------
%-------------------------------------------------------------------------------

\begin{document}

\setlength{\abovedisplayskip}{1pt}
\setlength{\belowdisplayskip}{1pt}



%-------------------------------------------------------------------------------
%-------------------------------------------------------------------------------
%-------------------------------------------------------------------------------
\insertTitle{Статистические методы анализа данных}{3}{ПМ-63}{Кожекин М.В.}{Майер В.А.}{Назарова Т.А.}{Утюганов Д.С.}{1}{2019}




%-------------------------------------------------------------------------------
%-------------------------------------------------------------------------------
%-------------------------------------------------------------------------------
\section{Цель работы}
Разработать программу для генерации экспериментальных данных по схеме имитационного моделирования и решить обратную задачу.




%-------------------------------------------------------------------------------
%-------------------------------------------------------------------------------
%-------------------------------------------------------------------------------
\section{Решение}

1.	Изменить модель регрессии, добавив в неё дополнительный регрессор, ранее не входивший в состав модели, порождающей данные.
Не генерируя новых данных, найти точечные оценки всех параметров расширенной модели. В дальнейшем при расмотрении этой расширенной
модели анализе должно быть показано, что параметр при дополнительном регрессоре незначим.

\inputThreeImages{../../1/pics/before_15_0_0.png}{../../1/pics/before_15_0_90.png}{../../1/pics/before_15_None_None.png}

\inputThreeImages{../../1/pics/after_15_0_0.png}{../../1/pics/after_15_0_90.png}{../../1/pics/after_15_None_None.png}


2,3. Построить доверительные интервалы для каждого параметра модели и проверить гипотезу о его значимости

Доверительный интервал для каждого параметра $\theta$ представим в виде двустороннего неравенства:

\(
\hat{\theta}_j - t_{\alha / 2, f_R} \leq \theta_j \leq \hat{\theta}_j + t_{\alha / 2, f_R}$
\)

Для определения значимости параметра необходимо посчитать статистику $F = \frac{ (\hat{\theta}_j)^2 }{ \sigma^2 d_{jj} }$ и её критическое значение $F_{\alpha,1,n-m}$

\inputTable{table_4_th_15.txt}

Как мы видим новый регрессор оказался незначим.




4. Проверить гипотезу о незначимости самой регрессии

\(
\begin{aligned}
F &= \frac{(RSS_H - RSS) / q}{RSS / (n - m)} \\
RSS &= {\left(y - X \hat{\theta} \right)}^T (y - X\hat{\theta}) \\
RSS_H &= \sum_{i=1}^n(y_i - \overline{y})^2 \\
q &= m - 1
\end{aligned}
\)


Гипотеза о незначимости регрессии отвергается, т.к. F = 21093.487 > F_{\alpha,q,n-m} = 2.699



5. Рассчитать прогнозные занчения для математического ожидания функции отклика $\eta(x,\hat{\theta}) = f^T (x) \hat{\theta}$
для всего интервала действия одного из факторов, зафиксировав значения других факторов на границе или в центре области их определения.

6. По полученным в п.5 прогнозным значениям построить графики прогнозных значений и доверительной полосы для математического ожидания функцииspa отклика и для самого отклика.

Для оценки математического ожидания функции отклика построим доверительный интервал:

\(
\eta(x, \hat{\theta}) - t_{\alha / 2, f_R} \sigma(\eta(x, \hat{\theta}))        \leq        \eta(x, \hat{\theta})        \leq        \eta(x, \hat{\theta}) + t_{\alha / 2, f_R} \sigma(\eta(x, \hat{\theta}))

\sigma(\eta(x, \hat{\theta})) = \hat{\sigma} \sqrt{1 + f^T(x)(X^TX)^{-1}f(x)}
\)

\inputTwoImages{../pics/conf_y_for_x1_15.png}{../pics/conf_y_for_x2_15.png}

7. Заново смоделировать исходные данные (см. лаб. работу №1), увеличив мощность случайной помехи до 50...70\% от мощности полезного сигнала, и провести оценку параметров. Повторить пункты 3, 4 с новыми данными.

\inputThreeImages{../../1/pics/before_70_0_0.png}{../../1/pics/before_70_0_90.png}{../../1/pics/before_70_None_None.png}

\inputThreeImages{../../1/pics/after_70_0_0.png}{../../1/pics/after_70_0_90.png}{../../1/pics/after_70_None_None.png}

\inputTable{table_4_th_70.txt}

Как видно, даже при таком высоком уровне шума модель адекватно оценивает параметры.

%-------------------------------------------------------------------------------
%-------------------------------------------------------------------------------
%-------------------------------------------------------------------------------
\section{Исходный код программы}

\myCodeInput{python}{run\_lab3.py}{../../run_lab3.py}

\myCodeInput{python}{model\_3parameters.py}{../../model_3parameters.py}
\myCodeInput{python}{model\_4parameters.py}{../../model_4parameters.py}

\myCodeInput{python}{lab1.py}{../../1/lab1.py}
\myCodeInput{python}{lab2.py}{../../2/lab2.py}
\myCodeInput{python}{lab3.py}{../lab3.py}


\end{document}
