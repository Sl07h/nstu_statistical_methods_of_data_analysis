\documentclass[12pt, a4paper]{article}
\usepackage[russian]{babel}
\usepackage{fontspec}
\setsansfont{Calibri}
\setmonofont{Consolas}
\setmainfont[
    Ligatures=TeX,
    Extension=.otf,
    BoldFont=cmunbx,
    ItalicFont=cmunti,
    BoldItalicFont=cmunbi,
]{cmunrm}
\usepackage{polyglossia}
\setdefaultlanguage{russian}
\setotherlanguage{english}


\usepackage{geometry}
\usepackage{pgfplotstable}
\usepackage[utf8]{inputenc}
\usepackage[english]{babel}
 
\setlength{\parindent}{0.7em}
\setlength{\parskip}{0.7em}
\geometry{
margin=2cm
}

% Создаем команду, чтобы переносить текст на новую строку внутри таблицы
\newcommand{\tcell}[2][l]{\begin{tabular}[#1]{@{}c@{}}#2\end{tabular}}

\usepackage{indentfirst}

\usepackage{arydshln}
\usepackage[fleqn]{amsmath}
\usepackage{xfrac}
\usepackage{esint}
\usepackage{amssymb}
\usepackage{mathbbol}
\usepackage[T1]{fontenc}
\usepackage{mathtools}
\usepackage{color}
\usepackage{ulem}
\usepackage{tabu}
\usepackage{multirow}
\usepackage{rotating}
\usepackage{enumitem}
%\usepackage{booktabs}
\usepackage[outline]{contour}
\contourlength{1.2pt}

\usepackage{tikz}
\usepackage{graphics}
\usepackage{xcolor}

\usepackage{pgfplots}
\usepackage{pgfplotstable}

\usepackage[at]{easylist}

\DeclareMathOperator{\sign}{sign}



%-------------------------------------------------------------------------------
%-------------------------------------------------------------------------------
%-------------------------------------------------------------------------------
\newcommand{\insertTitle}[9]{
\begin{titlepage}
	\begin{center}
    	\large
		Министерство науки и высшего образования Российской Федерации
		
		Новосибирский государственный технический университет
		
		Кафедра теоретической и прикладной информатики
		%\vspace{0.25cm}
		\vfill
		{\textbf #1}
		
		Лабораторная работа №#2
		\vfill
	\end{center}
	
	\begin{tabular}{ m{7em}  m{7em} }
	Факультет: & ФПМИ \\ 
	Группа: & #3 \\  
	Студенты: & #4 \\
			   & #5 \\
			   & #6 \\
			   & #7 \\	
	Вариант: & #8
	\end{tabular}
	\vfill

\begin{center}
Новосибирск

#9
\end{center}
\end{titlepage}
}




\newcommand{\inputTable}[1]{
\noindent{\pgfplotstabletypeset[
 	fixed,
    fixed zerofill,
    precision=3,
	columns/theta_l/.style	={column type={|c|}, column name={$\hat{\theta}_j - t_{\alha / 2, f_R} \sigma(\hat{\sigma}_j)$}},
	columns/theta/.style		={dec sep align={c|}, column type/.add={}{|}, column name={$\hat{\theta}_j$} },
	columns/theta_r/.style	={dec sep align={c|}, column type/.add={}{|}, column name={$\hat{\theta}_j + t_{\alha / 2, f_R} \sigma(\hat{\sigma}_j)$} },
	columns/F/.style			={dec sep align={c|}, column type/.add={}{|}, column name={$F$} },
	columns/F_t/.style		={dec sep align={c|}, column type/.add={}{|}, column name={$$F_{\alpha,q,n-m}$$} },
	columns/significance/.style	={string type, column type/.add={}{|}, column name={$\text{значимость параметра}$} },
	every head row/.style={before row=\hline,after row=\hline},
	every last row/.style={after row=\hline},
]{#1}}
}




\usepackage[utf8]{inputenc}
\usepackage{listings}
\usepackage{color}
 
\definecolor{codegreen}{rgb}{0,0.6,0}
\definecolor{codegray}{rgb}{0.5,0.5,0.5}
\definecolor{codepurple}{rgb}{0.58,0,0.82}
\definecolor{backcolour}{rgb}{1,1,1}
 
\lstdefinestyle{mystyle}{
	backgroundcolor=\color{backcolour},   
	commentstyle=\color{codegreen},
	keywordstyle=\color{blue},
	basicstyle=\fontsize{10}{12}\selectfont\ttfamily,
   	numberstyle=\tiny\color{codegray},
	stringstyle=\color{codepurple},
    	breakatwhitespace=false,         
	breaklines=true,                 
	captionpos=b,                    
	keepspaces=true,                 
	numbers=left,                    
	numbersep=5pt,                  
	showspaces=false,                
	showstringspaces=false,
	showtabs=false,                  
	tabsize=4
}
\lstset{ style=mystyle}


\newcommand{\inputTwoImages}[2]{
\begin{figure}[htbp!]
    \noindent
        \includegraphics[width=0.48\textwidth]{#1}
        \includegraphics[width=0.48\textwidth]{#2}
\end{figure}
}


\newcommand{\inputThreeImages}[3]{
\begin{figure}[htbp!]
    \noindent
        \includegraphics[width=0.32\textwidth]{#1}
        \includegraphics[width=0.32\textwidth]{#2}
        \includegraphics[width=0.32\textwidth]{#3}
\end{figure}
}



\newcommand{\myCodeInput}[3]{
{\bf Файл #2}
\lstinputlisting[language=#1]{#3}
}



%-------------------------------------------------------------------------------
%-------------------------------------------------------------------------------
%-------------------------------------------------------------------------------

\begin{document}

\setlength{\abovedisplayskip}{1pt}
\setlength{\belowdisplayskip}{1pt}



%-------------------------------------------------------------------------------
%-------------------------------------------------------------------------------
%-------------------------------------------------------------------------------
\insertTitle{Статистические методы анализа данных}{5}{ПМ-63}{Кожекин М.В.}{Майер В.А.}{Назарова Т.А.}{Утюганов Д.С.}{1}{2019}




%-------------------------------------------------------------------------------
%-------------------------------------------------------------------------------
%-------------------------------------------------------------------------------
\section{Цель работы}
Разработать программу для определения мультиколлинеарности модели.




%-------------------------------------------------------------------------------
%-------------------------------------------------------------------------------
%-------------------------------------------------------------------------------
\section{Решение}

1.	В соответствии с вариантом задания сгенерировать экспериментальные данные, в которых в явном виде присутствует эффект мультиколлинеарности.

Согласно заданию регерссия на 6 факторах. Эффект мультиколлинеарности создают 3 фактора. Имеется разброс в масштабах факторов.

\[ \eta = f^T(x) =  x_1 + x_2 + x_3 + x_4 + x_5 + x_6 \]
\[ x_1 \in [-1, 1] \]
\[ x_2 \in [-1, 1] \]
\[ x_3 \in [-1, 1] \]
\[ x_4 \in [-1, 1] \]
\[ x_5 \in [-1, 1] \]
\[ x_6  = 2 x_4 - 3 x_5 + N(0, 0.01) \]
\[ \rho = 0.15, n = 100, m = 6 \]




2. Рассчитать ряд показателей, характеризующих эффект мультиколлинеарности. Определить факторы, ответсвенные за возникновение  эффекта мультиколлинеарности.

В качестве мер измерения эффекта мультиколлинеарности рассмотрим следующее:

\begin{enumerate}

\item определитель информационной матрицы 

\[
|X^T X| = \prod_{i=1}^m \lambda_i = 7.22
\]

\item минимальное собственное число матрицы

\[
\lambda_1 = \lambda_{min}(X^T X) = 4.77e-06
\]

\item мера обусловленности матрицы по Нейману-Голдстейну

\[
\frac{\lambda_{max}(X^T X)}{\lambda_{min}(X^T X)} = 34699460.47
\]

\item максимальная парная сопряжённость

\[
\max\limits_{i,j}x|r_{ij}|, i \neq j 
\]
\[
R = (r_{ij}) - \text{матрица сопряжённости}
\]
\[
r_{ij} = cos(\underline{x}_i,\underline{x}_j) = \frac{(x_i, x_j)}{|x_i| |x_j|}
\]
\[
\max\limits_{i,j}x|r_{ij}| = 0.84
\]


\item максимальная сопряжённость

В качестве меры мультиколлинеарности возьмём 

\[
\max\limits_{i}|R_i|\text{, где}:
\]
\[
R_i = \sqrt{1 - \frac{1}{R_{ii}^{-1}}}, i=1,m
\]
\[
\max\limits_{i}|R_i| = 0.99
\]


\end{enumerate}


3. Построить ридж-оценки параметров при различных значениях параметров регуляризации. Выбрать оптимальное значение параметра регуляризации.
Построить графики изменения квадрата евклидовой нормы оценок параметров и остаточной суммы квадратов от параметра регуляризации.


Один из способов оценивания параметров в условиях мултиколлинеарности
 состоит в управлении масштабом полученных оценок. Однако это смещение значительно меньше, чем у обычных МНК оценок.
 
С целью управления масштабом оценок введём в расмотрение функцию стоимости

\[
C = \sum_{i=1}^{n}(y_i - f(X_i)\theta)^2 + \sum_{j=1}^{m}{\lambda_j \theta_j^2} 
\]

где второе слагаемое рассматривается как штраф при условии, что $\lambda_j \geq 0$

Минимизируя С, получим $\hat{\theta} = (X^TX + \Lambda)^{-1}X^Ty$, которые известны как ридж-оценки. Часто матрицу $\Lambda$ задают диагональной в виде $\Lambda_{ii} = \lambda (X^TX)_{ii}, \lambda \geq 0$.

Оптимальное значение $\lambda = m \hat{\sigma}^2 / ||\hat{\beta}||^2$

Графики:

\inputTwoImages{../pics/RSS_from_lambda.png}{../pics/theta_from_lambda.png}



4. Провести оценивание модели регрессии по методу главных компанентов. Перейти к описанию
в исходном пространстве факторов. Сравнить решение с ридж-оцениванием по смещению оценок 
и точности предсказания отклика.

Для начала мы центрируем нашу исходнную модель $$y_t = \Theta_0 + \Theta_1x_{1t}+...+\Theta_kx_{kt}+E_t$$

Тогда модель наблюдения будет иметь вид $$y_t^* = \beta_0 + \beta_1x_{1t}^*+...+\beta_kx_{kt}^*+E_t$$

Т.к. меру изменчивости можно измерить при помощи собственных значений, то мы можем получить матрицу главных компонент через матрицу $$V = (v_1,...,v_k)$$
$$Z = X^*V$$

После отбора главных компонент можно оценить регрессию $y^*$ на главные компоненты:
$$y_t^* = b_1 z_{1t} + ... + b_l z_{lt} + E_t$$
$$\hat{b} = (Z^TZ)^{-1} Z^T y^* $$



\[
\theta = 1, 1, 1, 1, 1, 1
\]
\[
\hat{\theta} = 1.78, 3.58, 0.56, 1.30, 1.13, 0.63
\]

%-------------------------------------------------------------------------------
%-------------------------------------------------------------------------------
%-------------------------------------------------------------------------------
\section{Текст программы}

\myCodeInput{python}{run\_lab5.py}{../../run_lab5.py}

\myCodeInput{python}{model\_6parameters.py}{../../model_6parameters.py}

\myCodeInput{python}{lab1.py}{../../1/lab1.py}
\myCodeInput{python}{lab5.py}{../lab5.py}


\end{document}
